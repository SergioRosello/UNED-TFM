\newglossaryentry{CSA}
{
        name=Container Security Analysis,
        description={Análisis de contenedores e imágenes en busca de
        vulnerabilidades de seguridad}
}
\newglossaryentry{SCA}
{
        name= Software Composition Analysis,
        description= {Es un proceso automatizado que identifica la existencia de
        librerías de software libre dentro de un proyecto. La finalidad de este
        análisis es evaluar la seguridad del proyecto, las licencias que se usan
        y la calidad del código del proyecto.}
}
\newglossaryentry{IAST}
{
        name=Interactive Application Security Testing,
        description={Analiza el código en busca de vulnerabilidades de seguridad
        mientras la aplicación está siendo revisada con tests automáticos,
        pruebas manuales o cualquier actividad que interactúe con la aplicación.}
}
\newglossaryentry{DAST}
{
        name=Dynamic Application Security Testing,
        description={Herramientas diseñadas para detectar condiciones que
        indiquen vulnerabilidades de seguridad en una aplicación mientras está
        activa.}
}
\newglossaryentry{SAST}
{
        name=Static Application Security Testing,
        description={Es una serie de tecnologías diseñadas para analizar el
        código fuente, bits y binarios de la aplicación en busca de indicadores
        de vulnerabilidades de seguridad. Las soluciones SAST analizan la
        aplicacion de dentro a afuera sin que estas estén en ejecución.  }
}
\newglossaryentry{job}
{
        name=job,
        description={Proceso ejecutado automáticamente, a partir de un evento externo, varios jobs ordenados de forma lógica forman un pipeline}
}
\newglossaryentry{API}
{
        name=Application Programming Interface, description={Permite que un
        servicio y un producto se comuniquen, permitiendo al producto crear una
        funcionalidad usando la información que proporciona el servicio a través
        de una interfaz estipulada.  Los desarrolladores no necesitan saber como
        se ha implementado el servicio, solamente, como se usa, para obtener la
        información requerida.}
}
\newglossaryentry{staging}
{
        name=staging,
        description={Un entorno que reproduce exactamente el entorno de producción cuya finalidad es asegurar que el código que se va a desplegar en producción funcione correctamente}
}
\newglossaryentry{App-Sec}
{
        name=Application Security,
        description={El dominio de la seguridad de la aplicación dentro del 
        sector de la seguridad.}
}
\newglossaryentry{DevOps}
{
        name=DevOps,
        description={Un nivel de abstracción superior al desarrollo ágil cuya
        finalidad es conseguir una sinergia entre personas, procesos y
        herramientas}
}
\newglossaryentry{STRIDE}
{
        name=STRIDE,
        description={Un acrónimo en lengua inglesa para las siguientes palabras:
        Spoofing, Tampering, Repudiation, Information Disclosure, Denial of
        service y Elevation of privilege}
}
\newglossaryentry{DevSecOps}
{
        name=DevSecOps,
        description={Una inclusion de las herramientas de seguridad en la
        filosofía DevOps}
}
\newglossaryentry{PenTest}
{
        name=PenTest,
        description={Una serie de revisiones manuales o automáticas que se
        realizan contra una pagina web o servicio expuesto públicamente en
        Internet cuya
        finalidad es vulnerar dicha pagina o servicio}
}
\newglossaryentry{pipeline}
{
        name=pipeline,
        description={Serie de procesos automatizados que realizan comprobaciones automatizadas sobre el programa que se pretende desplegar en el entorno de producción}
}

\newacronym{SDLC}{SDLC}{Software Development Life Cycle}
\newacronym{mr}{MR}{Merge Request}

%\newglossaryentry{<++>}
%{
        %name=<++>,
        %description={<++>}
%}
\newglossaryentry{EC2}
{
        name=Elastic Cloud Compute,
        description={es un servicio web que proporciona capacidad informática en la nube segura y de tamaño modificable.}
}
\newglossaryentry{merge}
{
        name=merge,
        description={Merging is Git's way of putting a forked history back together again.}
}
\newglossaryentry{monolito}
{
        name=monolito,
        description={hace referencia a una aplicación software en la que la capa
        de interfaz de usuario y la capa de acceso a datos están combinadas en
        un mismo programa y sobre una misma plataforma. }
}
\newglossaryentry{runner}
{
        name=runner,
        description={GitLab Runner is an application that works with GitLab
        CI/CD to run jobs in a pipeline.}
}
\newglossaryentry{coverage}
{
        name=coverage,
        description={Code coverage is a metric that can help you understand how
        much of your source is tested.}
}
\newglossaryentry{c-level}
{
        name=C-level,
        description={C-level executives play a strategic role within an
        organization; they hold senior positions and impact company-wide
        decisions}
}
\newglossaryentry{commit}
{
       name=commit,
       description={a commit is an operation which sends the latest changes of
       the source code to the repository, making these changes part of the head
       revision of the repository}
}
\newglossaryentry{PenTest}
{
        name=PenTest,
        description={A method of testing where testers target individual binary
        components or the application as a whole to determine whether intra or
        intercomponent vulnerabilities can be exploited to compromise the
        application, its data, or its environment resources.}
}
\newglossaryentry{SQLi}
{
        name=SQL Injection,
        description={A SQL injection attack consists of insertion or “injection”
        of a SQL query via the input data from the client to the application. A
        successful SQL injection exploit can read sensitive data from the
        database, modify database data (Insert/Update/Delete), execute
        administration operations on the database (such as shutdown the DBMS),
        recover the content of a given file present on the DBMS file system and
        in some cases issue commands to the operating system.}
}
\newglossaryentry{XSS}
{
        name=XSS,
        description={Cross-Site Scripting (XSS) attacks are a type of injection,
        in which malicious scripts are injected into otherwise benign and
        trusted websites. XSS attacks occur when an attacker uses a web
        application to send malicious code, generally in the form of a browser
        side script, to a different end user.}
}
\newglossaryentry{CSA}
{
        name=Container Security Analysis,
        description={Análisis de contenedores e imágenes en busca de
        vulnerabilidades de seguridad}
}
\newglossaryentry{SCA}
{
        name= Software Composition Analysis,
        description= {Es un proceso automatizado que identifica la existencia de
        librerías de software libre dentro de un proyecto. La finalidad de este
        análisis es evaluar la seguridad del proyecto, las licencias que se usan
        y la calidad del código del proyecto.}
}
\newglossaryentry{IAST}
{
        name=Interactive Application Security Testing,
        description={Analiza el código en busca de vulnerabilidades de seguridad
        mientras la aplicación está siendo revisada con tests automáticos,
        pruebas manuales o cualquier actividad que interactúe con la aplicación.}
}
\newglossaryentry{DAST}
{
        name=Dynamic Application Security Testing,
        description={Herramientas diseñadas para detectar condiciones que
        indiquen vulnerabilidades de seguridad en una aplicación mientras está
        activa.}
}
\newglossaryentry{SAST}
{
        name=Static Application Security Testing,
        description={Es una serie de tecnologías diseñadas para analizar el
        código fuente, bits y binarios de la aplicación en busca de indicadores
        de vulnerabilidades de seguridad. Las soluciones SAST analizan la
        aplicacion de dentro a afuera sin que estas estén en ejecución.  }
}
\newglossaryentry{job}
{
        name=job,
        description={Proceso ejecutado automáticamente, a partir de un evento
        externo, varios jobs ordenados de forma lógica forman un pipeline}
}
\newglossaryentry{API}
{
        name=Application Programming Interface, description={Permite que un
        servicio y un producto se comuniquen, permitiendo al producto crear una
        funcionalidad usando la información que proporciona el servicio a través
        de una interfaz estipulada.  Los desarrolladores no necesitan saber como
        se ha implementado el servicio, solamente, como se usa, para obtener la
        información requerida.}
}
\newglossaryentry{staging}
{
        name=staging,
        description={Un entorno que reproduce exactamente el entorno de
        producción cuya finalidad es asegurar que el código que se va a
        desplegar en producción funcione correctamente}
}
\newglossaryentry{App-Sec}
{
        name=Application Security,
        description={El dominio de la seguridad de la aplicación dentro del 
        sector de la seguridad.}
}
\newglossaryentry{DevOps}
{
        name=DevOps,
        description={Un nivel de abstracción superior al desarrollo ágil cuya
        finalidad es conseguir una sinergia entre personas, procesos y
        herramientas}
}
\newglossaryentry{STRIDE}
{
        name=STRIDE,
        description={Un acrónimo en lengua inglesa para las siguientes palabras:
        Spoofing, Tampering, Repudiation, Information Disclosure, Denial of
        service y Elevation of privilege}
}
\newglossaryentry{DevSecOps}
{
        name=DevSecOps,
        description={Una inclusion de las herramientas de seguridad en la
        filosofía DevOps}
}
\newglossaryentry{PenTest}
{
        name=PenTest,
        description={Una serie de revisiones manuales o automáticas que se
        realizan contra una pagina web o servicio expuesto públicamente en
        Internet cuya
        finalidad es vulnerar dicha pagina o servicio}
}
\newglossaryentry{pipeline}
{
        name=pipeline,
        description={Serie de procesos automatizados que realizan comprobaciones
        automatizadas sobre el programa que se pretende desplegar en el entorno
        de producción}
}
\newglossaryentry{JWT}
{
        name=JSON Web Token,
        description={JSON Web Token (JWT) is an open standard (RFC 7519) that
        defines a compact and self-contained way for securely transmitting
        information between parties as a JSON object. This information can be
        verified and trusted because it is digitally signed. JWTs can be signed
        using a secret (with the HMAC algorithm) or a public/private key pair
        using RSA or ECDSA.}
}

\newacronym{AWS}{AWS}{Amazon Web Services}
\newacronym{DTL}{DTL}{Development Team Leader}
\newacronym{MR}{MR}{Merge Request}
\newacronym{SLO}{SLO}{Service Level Objective}
\newacronym{HoT}{HoT}{Head of Technology}
\newacronym{PoC}{PoC}{Proof of Concept}
\newacronym{AppSec}{AppSec}{Application Security}
\newacronym{SRE}{SRE}{Site Reliability Engineer}
\newacronym{SDLC}{SDLC}{Software Development Life Cycle}
\newacronym{mr}{MR}{Merge Request}

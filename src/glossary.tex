\newglossaryentry{job}
{
        name=job,
        description={Proceso ejecutado automáticamente, a partir de un evento externo, varios jobs ordenados de forma lógica forman un pipeline}
}
\newglossaryentry{API}
{
        name=Application Programming Interface, description={Permite que un
        servicio y un producto se comuniquen, permitiendo al producto crear una
        funcionalidad usando la información que proporciona el servicio a través
        de una interfaz estipulada.  Los desarrolladores no necesitan saber como
        se ha implementado el servicio, solamente, como se usa, para obtener la
        información requerida.}
}
\newglossaryentry{staging}
{
        name=staging,
        description={Un entorno que reproduce exactamente el entorno de producción cuya finalidad es asegurar que el código que se va a desplegar en producción funcione correctamente}
}
\newglossaryentry{App-Sec}
{
        name=Application Security,
        description={El dominio de la seguridad de la aplicación dentro del 
        sector de la seguridad.}
}
\newglossaryentry{DevOps}
{
        name=DevOps,
        description={Un nivel de abstracción superior al desarrollo ágil cuya
        finalidad es conseguir una sinergia entre personas, procesos y
        herramientas}
}
\newglossaryentry{STRIDE}
{
        name=STRIDE,
        description={Un acrónimo en lengua inglesa para las siguientes palabras:
        Spoofing, Tampering, Repudiation, Information Disclosure, Denial of
        service y Elevation of privilege}
}
\newglossaryentry{DevSecOps}
{
        name=DevSecOps,
        description={Una inclusion de las herramientas de seguridad en la
        filosofía DevOps}
}
\newglossaryentry{PenTest}
{
        name=PenTest,
        description={Una serie de revisiones manuales o automáticas que se
        realizan contra una pagina web o servicio expuesto públicamente en
        Internet cuya
        finalidad es vulnerar dicha pagina o servicio}
}
\newglossaryentry{pipeline}
{
        name=pipeline,
        description={Serie de procesos automatizados que realizan comprobaciones automatizadas sobre el programa que se pretende desplegar en el entorno de producción}
}

\newacronym{SDLC}{SDLC}{Software Development Life Cycle}
\newacronym{mr}{MR}{Merge Request}

%----------
%   WARNING
%----------

% This Guide contains Library recommendations based mainly on APA and IEEE styles, but you must always follow the guidelines of your TFG Tutor and the TFG regulations for your degree.

% THIS TEMPLATE IS BASED ON THE APA STYLE 


%----------
% DOCUMENT SETTINGS
%----------

\documentclass[12pt]{report} % font: 12pt

% margins: 2.5 cm top and bottom; 3 cm left and right
\usepackage[
a4paper,
vmargin=2.5cm,
hmargin=3cm
]{geometry}

% Paragraph Spacing and Line Spacing: Narrow (6 pt / 1.15 spacing) or Moderate (6 pt / 1.5 spacing)
\renewcommand{\baselinestretch}{1.15}
\parskip=6pt

% Color settings for cover and code listings 
\usepackage[table]{xcolor}
\definecolor{azulUC3M}{RGB}{0,0,102}
\definecolor{gray97}{gray}{.97}
\definecolor{gray75}{gray}{.75}
\definecolor{gray45}{gray}{.45}

% In the template we include the file OUTPUT.XMPDATA. You can download that file and include the metadata that will be incorporated into the PDF file when you compile the memoria.tex file. Then upload it back to your project. 
% Commented out due to upstream error
% \use package[a-1b]{pdf}

% LINKS
\usepackage{hyperref}
\hypersetup{colorlinks=true,
	linkcolor=black, % links to parts of the document (e.g. index) in black
	urlcolor=blue} % links to resources outside the document in blue

% MATH EXPRESSIONS
\usepackage{amsmath,amssymb,amsfonts,amsthm}

% Character encoding
\usepackage{txfonts} 
\usepackage[T1]{fontenc}
\usepackage[utf8]{inputenc}

% English settings
\usepackage[spanish]{babel} 
\usepackage[babel, spanish=spanish]{csquotes}
\AtBeginEnvironment{quote}{\small}

% Footer settings
\usepackage{fancyhdr}
\pagestyle{fancy}
\fancyhf{}
\renewcommand{\headrulewidth}{0pt}
\rfoot{\thepage}
\fancypagestyle{plain}{\pagestyle{fancy}}

% DESIGN OF THE TITLES of the parts of the work (chapters and epigraphs or sub-chapters)
\usepackage{titlesec}
\usepackage{titletoc}
\titleformat{\chapter}[block]
{\large\bfseries\filcenter}
{\thechapter.}
{5pt}
{\MakeUppercase}
{}
\titlespacing{\chapter}{0pt}{0pt}{*3}
\titlecontents{chapter}
[0pt]                                               
{}
{\contentsmargin{0pt}\thecontentslabel.\enspace\uppercase}
{\contentsmargin{0pt}\uppercase}                        
{\titlerule*[.7pc]{.}\contentspage}                 

\titleformat{\section}
{\bfseries}
{\thesection.}
{5pt}
{}
\titlecontents{section}
[5pt]                                               
{}
{\contentsmargin{0pt}\thecontentslabel.\enspace}
{\contentsmargin{0pt}}
{\titlerule*[.7pc]{.}\contentspage}

\titleformat{\subsection}
{\normalsize\bfseries}
{\thesubsection.}
{5pt}
{}
\titlecontents{subsection}
[10pt]                                               
{}
{\contentsmargin{0pt}                          
	\thecontentslabel.\enspace}
{\contentsmargin{0pt}}                        
{\titlerule*[.7pc]{.}\contentspage}  


% Tables and figures settings
\usepackage{multirow} % combine cells 
\usepackage{caption} % customize the title of tables and figures
\usepackage{floatrow} % we use this package and its \ ttabbox and \ ffigbox macros to align the table and figure names according to the defined style.
\usepackage{array} % with this package we can define in the following line a new type of column for tables: custom width and centered content
\newcolumntype{P}[1]{>{\centering\arraybackslash}p{#1}}
\DeclareCaptionFormat{upper}{#1#2\uppercase{#3}\par}
\usepackage{graphicx}
\graphicspath{{../imagenes/}} % images folder

% Table layout for social sciences and humanities
\captionsetup*[table]{
	justification=raggedright,
	labelsep=newline,
	labelfont=small,
	singlelinecheck=false,
	labelfont=bf,
	font=small,
	textfont=it
}

% Figure layout for social sciences and humanities
\captionsetup[figure]{
	%name=Figura,
	singlelinecheck=off,
	labelsep=newline,
	font=small,
	labelfont=bf,
	textfont=it
}
\floatsetup[figure]{
    style=plaintop,
    heightadjust=caption,
    footposition=bottom,
    font=small
}

% Figures and tables footnote layout 
\captionsetup*[floatfoot]{
    footfont={small, up}
}

% FOOTNOTES
\usepackage{chngcntr} % continuous numbering of footnotes
\counterwithout{footnote}{chapter}

% CODE LISTINGS 
% support and styling for listings. More information in  https://es.wikibooks.org/wiki/Manual_de_LaTeX/Listados_de_código/Listados_con_listings
\usepackage{listings}

% Custom listing
\lstdefinestyle{estilo}{ frame=Ltb,
	framerule=0pt,
	aboveskip=0.5cm,
	framextopmargin=3pt,
	framexbottommargin=3pt,
	framexleftmargin=0.4cm,
	framesep=0pt,
	rulesep=.4pt,
	backgroundcolor=\color{gray97},
	rulesepcolor=\color{black},
	%
	basicstyle=\ttfamily\footnotesize,
	keywordstyle=\bfseries,
	stringstyle=\ttfamily,
	showstringspaces = false,
	commentstyle=\color{gray45},     
	%
	numbers=left,
	numbersep=15pt,
	numberstyle=\tiny,
	numberfirstline = false,
	breaklines=true,
	xleftmargin=\parindent
}

\captionsetup*[lstlisting]{font=small, labelsep=period}
 
\lstset{style=estilo}
\renewcommand{\lstlistingname}{\uppercase{Código}}


% REFERENCES 

\usepackage{color}

% used for cite command, and changing the color the URL and dates are.
\usepackage{hyperref} 
\hypersetup{
    colorlinks=true,
    linkcolor=blue,
    urlcolor=blue,
    linktoc=all,
    citecolor=black
           }

% APA bibliography setup
\usepackage[style=apa, backend=biber, natbib=true, hyperref=true, uniquelist=false, sortcites]{biblatex}

\addbibresource{referencias.bib} % The references.bib file in which the bibliography used should be

% Caption package, for use of subfigures.
\usepackage{subfig}

% Create a list of glossaries and acronyms
\usepackage[acronym]{glossaries}
\makeglossaries

%-------------
%	DOCUMENT
%-------------

\begin{document}

\pagenumbering{roman} % Roman numerals are used in the numbering of the pages preceding the body of the work.
	
%----------
%	COVER
%----------	
\begin{titlepage}
	\begin{sffamily}
  \begin{figure}%
    \raggedleft
    \subfloat{{\includegraphics[width=3cm]{UNED.jpg} }}%
    \hspace*{\fill}
    \subfloat{{\includegraphics[width=6cm]{Scalefast.jpg} }}%
\end{figure}
	\begin{center}
		\vspace{2.5cm}
		\begin{Large}
			Master en Ciberseguridad\\			
			 2020-2021\\
			\vspace{2cm}		
			\textsl{Trabajo de final de Master}
			\bigskip
			
		\end{Large}
		 	{\Huge ``Definición y puesta en funcionamiento del sistema operacional de seguridad en el ciclo de desarrollo/despliegue (DevSecOps)''}\\
		 	\vspace*{0.5cm}
	 		\rule{10.5cm}{0.1mm}\\
			\vspace*{0.9cm}
			{\LARGE Sergio Roselló Morell}\\ 
			\vspace*{1cm}
		\begin{Large}
			Rafael Pastor Vargas\\
			David Aracil Cofrade\\
			Madrid, a \today \\
		\end{Large}
	\end{center}
	\vfill
	\color{black}
	\fbox{
	\begin{minipage}{\linewidth}
    	\textbf{AVOID PLAGIARISM}\\
    	\footnotesize{The University uses the \textbf{Turnitin Feedback Studio} for the delivery of student work. This program compares the originality of the work delivered by each student with millions of electronic resources and detects those parts of the text that are copied and pasted. Plagiarizing in a TFM is considered a  \textbf{\underline{Serious Misconduct}}, and may result in permanent expulsion from the University.}\end{minipage}}

	% IF OUR WORK IS TO BE PUBLISHED UNDER A CREATIVE COMMONS LICENSE, INCLUDE THESE LINES. IS THE RECOMMENDED OPTION.
	\noindent\includegraphics[width=4.2cm]{creativecommons.png}\\ % Creative Commons Logo
    \footnotesize{This work is licensed under Creative Commons \textbf{Attribution – Non Commercial – Non Derivatives}}
	
	\end{sffamily}
\end{titlepage}

\newpage % blank page
\thispagestyle{empty}
\mbox{}




%----------
%	ABSTRACT AND KEYWORDS 
%----------	
\renewcommand\abstractname{\large\bfseries\filcenter\uppercase{Sumario}}
\begin{abstract}
\thispagestyle{plain}
\setcounter{page}{3}
	
Inicio del movimiento DevOps, el 2010, con el desarrollo agil\\
Adopcion de la cultura agil por silicon valley y resto de munto\\
Implementacion de seguridad dentro del desarrollo agil\\
Uso de Pipelines, para desarrollar SW con la seguridad en mente\\
Caso especifico de la implementacion en Scalefast\\
\vfill
\end{abstract}
	
\newpage % Blank page
\renewcommand\abstractname{\large\bfseries\filcenter\uppercase{Abstract}}
\begin{abstract}
\thispagestyle{plain}
\setcounter{page}{5}

Start of Agile SW development, and DevOps culture\\
Adoption of culture by Silicon Valley and subsequent popularization\\
Security inside agile development?\\
Use of Pipelines in security-focused Agile development\\
Specific case with Scalefast\\

\textbf{Palabras clave:} % add the keywords
	
\vfill
\end{abstract}
\newpage % Blank page
\thispagestyle{empty}
\mbox{}


%----------
%	Dedication
%----------	
\chapter*{Agradecimientos}

\setcounter{page}{7}
	
Agradecer principalmente este trabajo a:

\begin{itemize}
  \item{Familia}
  \item{Tutores}
  \item{UNED}
  \item{Scalefast}
\end{itemize}

		
	\vfill
	
	\newpage % blank page
	\thispagestyle{empty}
	\mbox{}
	




%----------
%	TOC
%----------	

%--
% TOC
%-
\tableofcontents
\thispagestyle{fancy}

\newpage % blank page
\thispagestyle{empty}
\mbox{}




%--
% List of figures. If they are not included, comment the following lines
%-
\listoffigures
\thispagestyle{fancy}

\newpage % blank page
\thispagestyle{empty}
\mbox{}




%--
% List of tables. If they are not included, comment the following lines
%-
\listoftables
\thispagestyle{fancy}

\newpage % blankpage
\thispagestyle{empty}
\mbox{}





\newglossaryentry{CSA}
{
        name=Container Security Analysis,
        description={Análisis de contenedores e imágenes en busca de
        vulnerabilidades de seguridad}
}
\newglossaryentry{SCA}
{
        name= Software Composition Analysis,
        description= {Es un proceso automatizado que identifica la existencia de
        librerías de software libre dentro de un proyecto. La finalidad de este
        análisis es evaluar la seguridad del proyecto, las licencias que se usan
        y la calidad del código del proyecto.}
}
\newglossaryentry{IAST}
{
        name=Interactive Application Security Testing,
        description={Analiza el código en busca de vulnerabilidades de seguridad
        mientras la aplicación está siendo revisada con tests automáticos,
        pruebas manuales o cualquier actividad que interactúe con la aplicación.}
}
\newglossaryentry{DAST}
{
        name=Dynamic Application Security Testing,
        description={Herramientas diseñadas para detectar condiciones que
        indiquen vulnerabilidades de seguridad en una aplicación mientras está
        activa.}
}
\newglossaryentry{SAST}
{
        name=Static Application Security Testing,
        description={Es una serie de tecnologías diseñadas para analizar el
        código fuente, bits y binarios de la aplicación en busca de indicadores
        de vulnerabilidades de seguridad. Las soluciones SAST analizan la
        aplicacion de dentro a afuera sin que estas estén en ejecución.  }
}
\newglossaryentry{job}
{
        name=job,
        description={Proceso ejecutado automáticamente, a partir de un evento externo, varios jobs ordenados de forma lógica forman un pipeline}
}
\newglossaryentry{API}
{
        name=Application Programming Interface, description={Permite que un
        servicio y un producto se comuniquen, permitiendo al producto crear una
        funcionalidad usando la información que proporciona el servicio a través
        de una interfaz estipulada.  Los desarrolladores no necesitan saber como
        se ha implementado el servicio, solamente, como se usa, para obtener la
        información requerida.}
}
\newglossaryentry{staging}
{
        name=staging,
        description={Un entorno que reproduce exactamente el entorno de producción cuya finalidad es asegurar que el código que se va a desplegar en producción funcione correctamente}
}
\newglossaryentry{App-Sec}
{
        name=Application Security,
        description={El dominio de la seguridad de la aplicación dentro del 
        sector de la seguridad.}
}
\newglossaryentry{DevOps}
{
        name=DevOps,
        description={Un nivel de abstracción superior al desarrollo ágil cuya
        finalidad es conseguir una sinergia entre personas, procesos y
        herramientas}
}
\newglossaryentry{STRIDE}
{
        name=STRIDE,
        description={Un acrónimo en lengua inglesa para las siguientes palabras:
        Spoofing, Tampering, Repudiation, Information Disclosure, Denial of
        service y Elevation of privilege}
}
\newglossaryentry{DevSecOps}
{
        name=DevSecOps,
        description={Una inclusion de las herramientas de seguridad en la
        filosofía DevOps}
}
\newglossaryentry{PenTest}
{
        name=PenTest,
        description={Una serie de revisiones manuales o automáticas que se
        realizan contra una pagina web o servicio expuesto públicamente en
        Internet cuya
        finalidad es vulnerar dicha pagina o servicio}
}
\newglossaryentry{pipeline}
{
        name=pipeline,
        description={Serie de procesos automatizados que realizan comprobaciones automatizadas sobre el programa que se pretende desplegar en el entorno de producción}
}

\newacronym{SDLC}{SDLC}{Software Development Life Cycle}
\newacronym{mr}{MR}{Merge Request}


%----------
%	THESIS
%----------	
\clearpage
\pagenumbering{arabic} % numbering with Arabic numerals for the rest of the document.	


\chapter{Metodología}

% EXPLICAR POR QUE HE ELEGIDO ESTE TEMA 

La razón que me llevo a la elección del tema \textit{Definición y puesta en 
funcionamiento del sistema operacional de seguridad en el ciclo de 
desarrollo/despliegue (DevSecOps)} reside en mi deseo por unir dos de los
sectores del mundo del software, que hasta hace poco, no han estado muy 
interrelacionados.
La seguridad, y la metodología de desarrollo ágil.

Mi carrera profesional es comparativamente corta, ya que tengo un par de años
como ingeniero QA y recientemente, estoy trabajando como DevOps en Scalefast. 
Aunque lleve poco tiempo en el sector empresarial del desarrollo de software,
pienso que he tenido suficiente tiempo, o suficiente suerte como para observar
ciertos aspectos de mis responsabilidades laborales que me han ido interesando
cada vez mas.

Una de las observaciones que he hecho sobre mi mismo durante mi experiencia en
este mundo es que el aprendizaje continuo es uno de mis intereses mas grandes.
Esta es la razón principal por la que me he inscrito al master de 
Ciberseguridad con la UNED mientras llevaba una vida laboral activa.  
Durante el estudio de las distintas asignaturas que forman el master, una de 
las que más ha estimulado mi creatividad y aspiraciones futuras ha sido
Auditoria y monitorización de la seguridad. 
Durante el tiempo que he dedicado a estudiar dicha asignatura, he aprendido que
garantizar que tu aplicación es segura, conlleva un tratado de
datos y registros masivo, y que para poder monitorizar dichos registros, hacen
falta tanto el dominio de las herramientas disponibles, como un conocimiento
profundo de la propia aplicacion sobre la que trabajas.

Para poder explicar el motivo de este trabajo, antes debo poner al lector en
contexto explicando el sector de negocio en el que se encuentra Scalefast y la
necesidad que pretende cubrir con el puesto de \gls{DevOps}.
Scalefast es una empresa que proporciona a empresas una plataforma desde la que
pueden basar toda su estrategia de venta on-line.
Recientemente, muchas empresas que se centraban en venta directa a otras
empresas, se han querido especializar en la venta directa al consumidor.
Este es el motivo por el cual Scalefast ha crecido drásticamente en
popularidad, y las tareas que antes eran gestionables entre pocos, ahora deben
ser automatizadas para poder seguir el ritmo al que el mercado la hace crecer.
En vista de esta necesidad, se crearon varios puestos de \gls{DevOps}, uno de
los cuales, he sido afortunado de cubrir.
%TODO: Cual era la intencion con el puesto de DevOps? Por que hacia falta?

La reciente experiencia como DevOps en Scalefast me ha permitido formar parte
de áreas del ciclo de vida del desarrollo que únicamente había visto desde la
lejanía y que solo conocía teóricamente.
Gracias a la exposición a estas areas, me he dado cuenta de que mi interés es superior al esperado.
Una de las tareas en las que me he visto involucrado que mas me ha interesado 
ha sido la definición y análisis del estado del Pipeline y la búsqueda de su
evolución para garantizar un desarrollo mas ágil y seguro dentro de la
organización.
Dicho esto, puedo concluir que en este proyecto pretendo unir dos de mis intereses dentro del universo del software.  
La metodología ágil y su vertiente DevOps, con la seguridad.

% OBJETIVO PRINCIPAL DEL TRABAJO - Explorar la union entre los procesos de
% desarrollo que desencadenan en una construcción de software de calidad
% teniendo siempre en cuenta la seguridad.
Dada la situación en la que me encuentro laboralmente, pienso que estoy en una 
posición única para poder observar gran parte del ciclo de vida del software.
Una de las partes fundamentales de el DevOps es fomentar la comunicación entre
los distintos departamentos que forman parte del ciclo de vida del desarrollo.
Ya sea mediante reuniones presenciales de definición de procesos, como
implementando tareas (jobs) automatizadas que forman parte de un pipeline cuya
utilidad e intención es agilizar el proceso de desarrollo del software.
Ademas de la responsabilidad mencionada anteriormente, otra parte fundamental
que debemos desempeñar los DevOps es contagiar a los desarrolladores una
cultura de automatismo, desarrollo ágil, y responsabilidad sobre su código.  
De este modo, y teniendo en cuenta siempre la meta a la que se quiere llegar
como DevOps en Scalefast, que es proporcionar a los desarrolladores las
herramientas que necesitan para asegurar una calidad del software propia de 
Scalefast, ademas de agilizar los procesos existentes en nuestro \gls{SDLC}, el
objetivo de este trabajo es llevar a cabo el proceso de transición de un
modelo de integración continua y despliegue manual, en el que la seguridad
entra en juego en la última fase del \gls{SDLC}, a un modelo \gls{DevSecOps}
moderno, en el que la seguridad del desarrollo se transfiere a las primeras
fases del desarrollo, agilizando y asegurando de esta forma el desarrollo, en 
el que la integración y despliegue se automatizan hasta el momento en el que se
pretende sacar el software al entorno de producción.

Como estudiante del master de ciberseguridad, he de responder al creciente
interés en el mundo del desarrollo ágil por promover el desarrollo de software
de forma segura desde el inicio del ciclo de vida de la aplicación.
Esta evolución del movimiento DevOps, se denomina DevSecOps.
 
La situación el la que me encuentro me permite aplicar los conocimientos
adquiridos en el Master de forma práctica en el mudo laboral real.
Implementando la parte de \gls{App-Sec} al ciclo de vida de desarrollo de
software.
La filosofía de desarrollo que lleva a cabo la ejecución de dichas
responsabilidades es denominada comúnmente como \gls{DevSecOps}.

%TODO: 3. Escribir una descripción de como esta estructurado el documento, a
%nivel de capitulo, con referencias. (Pug parlar en primera persona) y citar las referencias (para poder llevar a cabo esta sección, ha sido necesaria la
%aportación de X)

Llegados a este punto creo que es buen momento para hablar de la forma en la 
que está estructurado este documento.
El propósito del mismo en el primer capítulo es proporcionar al lector una
visión inclusiva de los procesos empleados para la construcción del software y
de su evolución a través de los años.
Para llevar a cabo esta parte, hago uso del brillante articulo escrito por 
Craig Larman y Victor R. Basili para la IEEE Computer Society en junio de 2003,
en el que hacen un resumen de las diferentes formas en las que se ha 
desarrollado software hasta la actualidad.

Una vez llegamos a la actualidad, me parece esencial describir el desarrollo
ágil, y destacar que este no es una invención de Silicon Valley, y que este
movimiento o forma de desarrollo no nace de la nada, sino que evoluciona de
forma iterativa e incremental a medida que se van identificando las carencias 
en la forma de desarrollo y añadiendo remedios a estas.
Después de definir que es el desarrollo ágil como lo conocemos en la actualidad,
pasamos a describir una de las formas en la que se podría llevar a cabo este
desarrollo. 

Hablar de desarrollo ágil en la actualidad desencadena
inevitablemente en hablar de la filosofía \gls{DevOps} y \gls{DevSecOps}.
Me parece vital que queden claros estos conceptos e intento definirlos, siendo
consciente de la necesidad del movimiento de no ser reducido a una definición
común.
Esto es debido a que, al ser una idea abstracta, cada organización que quiera
implementar la filosofía DevOps, va a tener que hacerlo ajustándose a su propia
cultura, sus propios procesos y su propio ritmo.
En la siguiente sección analizo la primera, para luego dar paso al análisis de
la filosofía \gls{DevSecOps}.

En la segunda parte de este documento pretendo mostrar el proceso que se ha
seguido en Scalefast para transformar el pipeline actual, basado en Integración
Continua, a un \gls{pipeline} que es útil en una mayor área del ciclo de vida
de un proyecto, desde integración a despliegue continuos, sin dejar de lado el
desarrollo de software seguro.

La idea que se ha seguido a la hora de escribir esta sección es definir las
características idílicas de un \gls{SDLC} que guardan relación con \gls{App-Sec}.
Una vez tengamos claras cuales son, ir concretando los requisitos necesarios
para realizar cada una de las tareas.
Esto implica calcular el tiempo que vamos a tardar en hacer todas las tareas, 
si se va a necesitar contratar a expertos que nos asesoren en las nuevas tareas
, o que tareas se van a dejar de lado debido a incompatibilidades con el estado
actual de la tecnología con la que trabajamos.
Para llevar a termino el proyecto, se empieza creando una documentación 
gráfica, sobre la que nos vamos a apoyar y una documentación mas extensa en la
que podemos obtener información concreta, por ejemplo, de las distintas 
opciones en las herramientas que podemos usar para realizar un análisis 
estático del código.
%TODO: Escribir el nombre de la sección en la que se crea la tabla 
Esta parte queda descrita en la sección bajo el nombre <++>. 

%TODO: Asegurar que es lo que vamos a hacer después, porque no lo se aun
% Hasta aquí puedo escribir ahora, porque no he llagado mas lejos.
% lo demás sera mentira, y seguramente se tenga que cambiar
Seguidamente, creamos una estrategia de desarrollo, representada en forma de
proyecto. <++>
Del pipeline que se quiere implementar, se obtienen la cantidad de gente que se
necesita contratar para llevar la construcción y el mantenimiento de las
aflicciones que se van a integrar en el ciclo de vida del proyecto.

Si nos damos cuenta, la propia construcción del pipeline queda representada en
un mismo pipeline.
La única diferencia, es que el cliente en este caso es interno, y no externo.

Desde su inicio, en el que se definen las características de la aplicacion,
hasta la fase de monitorización, en la que se revisa continuamente el estado 
del proyecto para garantizar su correcto funcionamiento.
Crear este análisis del estado al que queremos llegar es importante porque nos
proporciona una idea global de los pasos que tenemos que seguir y nos permite
desarrollar una estrategia para llevar a cabo dicha estrategia.
%TODO: Por que es importante esto?

Al acabar de definir el \gls{pipeline} ideal, se pretende adaptarlo a las
limitaciones técnicas, económicas, o estratégicas de la empresa y 
posteriormente definir cuales son las tareas que podemos llevar a cabo y cuales
las que no podemos llevar a cabo y prescindir de ellas.

En la siguiente sección, se pretende describir el proceso de desarrollo de una
de estas tareas.
Pasando así por las fases del ciclo de vida del desarrollo descritas en la sección anterior.

Es en el tercer capitulo donde describo las conclusiones obtenidas a lo largo 
de este trabajo.

%TODO: Definir bien la
% estructura del documento antes de continuar.

\chapter{Estado del arte}

\section{Evolución del desarrollo de software, desde su concepción hasta la
actualidad}

Hablar del desarrollo del software desde su concepción hasta su actualidad
require especial atención ya que el software solo es posible gracias al hardware
sobre el que se ejecuta.  Naturalmente, este campo ha adoptado las costumbres y
características del mundo del Hardware.

Para entender la evolución del software, nos tenemos que remontar a los años
cincuenta de la década pasada, en la que mantener un ordenador suponía un gasto
enorme debido a que el software que corría en el ordenador estaba tan acoplado
al hardware sobre el que corría, que cada ordenador tenia su propio lenguaje de
programación.  No fue hasta 1957 que se desarrollo FORTRAN, el primer lenguaje
de programación, creando así el primer estándar de lenguaje de programación.
FORTRAN 66 \cite{FORTRAN1966}.  Al no disponer de una conexión a Internet y
requerir la presencia de ingenieros enviados por parte de la compañía que
fabrica el propio ordenador, las empresas que poseían un ordenador, no esperaban
tener que actualizar su software regularmente.  A medida que se avanzaba en la
fabricación de ordenadores personales y se estandarizaba Internet, los
requisitos para poder actualizar el software del ordenador empezaban a ofrecer
menos resistencia.

Pasamos del inicio de los lenguajes de programación al inicio de los comercios
en Internet.  A medida que evoluciona Internet, también lo hacen las
aplicaciones que se alojan en el, pasamos de Internet 1.0, a  Internet 2.0, en
el que las paginas web dejan de ser portales de visita, como revistas, y se
empieza a poder diferenciar clientes, y adaptar el funcionamiento de la pagina
al cliente especifico.  Estos nuevos requisitos, incorporan nueva complejidad a
la ya creciente complejidad del software, como por ejemplo, cualquier pagina que
tenga que poder diferenciar a sus clientes, por lógica de negocio, debe añadir a
su pila de componentes, una base de datos.

Debido a esta creciente complejidad, las empresas cuyo modelo de negocio estaba
basado en el software, ofrecían actualizaciones muy poco regularmente ya que a
mayor complejidad de aplicación, mayor era el tiempo que se tenia que designar a
la compilación de las distintas piezas que componían el software. (Base de
datos, Portal frontal, Sistema de lógica interna, entre otras.) Es por esto que
el proceso de Despliegue ha sido y sigue siendo uno de los puntos de estrés de
muchas empresas.

La realidad, es que desde 1957, en el que se pretendía desarrollar
iterativamente e incrementalmente \cite{IID}, se ha estado pensando
colectivamente en como mejorar el desarrollo de software, desde su fase de
concepción, a su fase de despliegue.  No obstante, no es hasta 2001 cuando se
escribió el manifesto del desarrollo ágil \cite{agile}, en el que se destacan
los pilares fundamentales sobre los que se ha construido la filosofía DevOps
\cite{CD-TF}, posteriormente extendida a la filosofía DevSecOps.

En conclusion, saber a cerca de la evolución del software facilita también la
comprensión de los objetivos de este trabajo que se resumen en: <++> %TODO: List
TFG objectives.


\subsection{Historia de los despliegues}

En esta sección analizaremos las distintas formas en las que se ha desarrollado
software comúnmente, hasta hoy en día, haciendo énfasis en el desarrollo ágil, y
el uso de \Gls{pipeline}s para garantizar cierta calidad del software antes de
desplegar.

Así pues, en la década de 1950, se empieza a pensar y desarrollar una mejor
forma de construir software.  El proyecto más memorable en el que se hace uso de
técnicas modernas de desarrollo de software es "Project Mercury".  Este
proyecto, centrado en poner al hombre a la luna, requería de una capacidad de
construcción de software ejemplar.  Gerald M. Weinberg fue el arquitecto del
proyecto.  Decidieron que ``el desarrollo en cascada aplicado a un proyecto
grande era bastante estúpido, o al menos, ignorante a la realidad'' \cite{GW-PM}
y desarrollaron software con una metodología similar a la estipulada por XP.
\cite{XP}

No es sin embargo hasta 1968 cuando encontramos la primera aparición formal de
desarrollo iterativo, que aparece en un documento interno de IBM en el que se
describe que primero se define formalmente la funcionalidad, posteriormente se
establecen unos tests, y se empieza a desarrollar, a medida que avanza el
proceso, se crean nuevos tests, mas específicos, al final, el sistema, se
convierte en la aplicación. \cite{ID-FB}

En los años setenta, se documenta por primera vez el desarrollo en cascada en el
articulo que escribe el Dr. Winston W. Royce.  Como curiosidad, Royce aconseja
realizar el proceso de desarrollo compuesto por las fases de análisis de
requisitos, construcción del software, testeo y despliegue dos veces, pero a
medida que se ha extendido esta metodología, se ha popularizado siguiendo
únicamente una de estas iteraciones, quedando en el desarrollo en cascada que se
conoce hoy en día. \cite{royce1970} Esta metodología adaptada, heredada del
articulo del Dr. Winston W. Royce sobre como desarrollar software, se popularizo
enormemente, a pesar de sus desventajas en cuanto a flexibilidad, adaptabilidad
y estimación de tiempos.

Otro notable proyecto que describe la forma en la que se han llevado a cabo sus
fases de desarrollo de software, es el desarrollo de una familia de compiladores
extendidos para una familia de lenguajes de programación específicos a dominio.
En este articulo, se describe claramente el desarrollo iterativo incremental
(IID). \cite{6312870}

Durante los años 70, el desarrollo en cascada cobra una popularidad abismal,
tanta, que los lideres de equipo encargados de proyectos tan importantes como la
creación del software de la lanzadera espacial de la NASA, sentían la obligación
de justificar por que no habían usado el método de de desarrollo en cascada.  La
justificación mas común para no usar el desarrollo en cascada, era que los
requisitos del proyecto cambiaban continuamente, y no podían ajustar el
desarrollo en cascada a estos cambios.

En los años 80, muchos desarrolladores prominentes publican artículos sobre como
el desarrollo en cascada no es una buena forma de construir un proyecto
medianamente grande de software.  Parafraseando el articulo titulado ``Un
proceso de diseño racional: Como y por que falsearlo'' \cite{Parnas1986}
\begin{itemize} \item{Un usuario raramente sabe exactamente todo lo que quiere y
    no puede expresar todo lo que sabe} \item{Existen muchos detalles de
      implementación que no se pueden anticipar, aunque tengamos todos los
      requisitos claros} \item{Aunque sepamos todos los detalles, como humanos,
      no podemos procesar tanta complejidad} \item{Aunque pudiésemos procesar
        toda esta complejidad, existen fuerzas exteriores, que hacen que cambien
        los requisitos o invaliden decisiones previas} \end{itemize} Estas
        razones encabezan la lista de motivos por los que la tendencia a la hora
        de desarrollar software debía de virar hacia el desarrollo incremental e
        iterativo.

Desde los años 90 hasta la actualidad, a medida que el desarrollo en cascada
demostraba que no era lo suficientemente flexible para llevar a cabo proyectos
medianamente grandes, cantidad de nuevos artículos aparecían con nuevas y
mejoradas técnicas de desarrollo iterativo e incremental.  Muchas de estas
siguen aplicándose hoy en día, como XP (Extreme Programming), Dynamic Systems
Development Method (DSDM), Scrum y FDD (Feature Driven Development), entre
otras.  El 1 de febrero de 2001, se celebra una reunion entre varios expertos en
procesos, entre ellos, los promotores de XP, Scrum, FDD.  En ese momento se
forma la alianza ágil (agile alliance), cuya función es promover métodos de
desarrollo de software iterativo e incremental. 

\section{Desarrollo ágil}

Como se ha reverenciado anteriormente, el desarrollo ágil forma parte del ADAN
del desarrollo de software.  Muchos de los proyectos mas complejos se han
construido usando técnicas que recuerdan mucho a la metodología ágil.
\cite{GW-PM} Afortunadamente, a medida que nos adentramos en la actualidad,
podemos observar como el uso de la metodología ágil aumenta en popularidad
\cite{Hoyada}.  A día de hoy, el 95\% de las empresas dicen desarrollar software
siguiendo la metodología ágil. \cite{stateofagile}

\subsection{Que es el desarrollo ágil}


El termino ``Desarrollo Ágil'' se acuñó en 2001, en una reunion que celebraron
distintos representares de alternativas al popular desarrollo en cascada.  En
esta reunion, asistieron representares de técnicas de desarrollo como Extreme
Programming, SCRUM, DSDM, Adaptive Software Development, Crystal, Feature-Driven
Development, Pragmatic Programming y otros simpatizantes.  La finalidad era
asentar unas bases sobre las que todos estuviesen de acuerdo.

Al terminar, se acuñó el término desarrollo ágil.

Los principales valores del desarrollo ágil son los siguientes:
\cite{agilePrinciples}

\begin{itemize} 
  \item{Nuestra mayor prioridad es satisfacer al cliente mediante la entrega 
        temprana y continua de software con valor.} 
  \item{Aceptamos que los requisitos cambien, incluso en etapas tardías del 
        desarrollo. Los procesos Ágiles aprovechan el cambio para proporcionar 
        ventaja competitiva al cliente.} 
  \item{Entregamos software funcional frecuentemente, entre dos semanas y dos 
        meses, con preferencia al periodo de tiempo más corto posible.} 
  \item{Los responsables de negocio y los desarrolladores trabajamos juntos de 
        forma cotidiana durante todo el proyecto.} 
  \item{Los proyectos se desarrollan en torno a individuos motivados. Hay que 
        darles el entorno y el apoyo que necesitan, y confiarles la ejecución 
        del trabajo.} 
  \item{El método más eficiente y efectivo de comunicar información al equipo de
        desarrollo y entre sus miembros es la conversación cara a cara.} 
  \item{El software funcionando es la medida principal de progreso.}
  \item{Los procesos Ágiles promueven el desarrollo sostenible. Los promotores,
        desarrolladores y usuarios debemos ser capaces de mantener un ritmo
        constante de forma indefinida.} 
  \item{La atención continua a la excelencia técnica y al buen diseño mejora la 
        Agilidad.} 
  \item{La simplicidad, o el arte de maximizar la cantidad de trabajo no 
        realizado, es esencial.} 
  \item{Las mejores arquitecturas, requisitos y diseños emergen de equipos 
        auto-organizados.} 
  \item{A intervalos regulares el equipo reflexiona sobre cómo ser más efectivo 
        para a continuación ajustar y perfeccionar su comportamiento en 
        consecuencia.} 
\end{itemize}

\subsection{Como llevar a cabo el desarrollo ágil}

Una vez repasados los valores del desarrollo ágil, debemos ahondar más en como
repercuten estos en el día a día de un proyecto de software que usa la
metodología ágil.

Cada uno de los pasos descritos a continuación forma parte de una iteración
dentro del proceso de desarrollo de software.  Estas fases, van construyendo
sobre las previas, hasta que se completa la aplicación.

\subsubsection{Firma del contrato}

Al inicio de la colaboración, se tiene una reunion, en la que el cliente y la
empresa trabajan conjuntamente para asentar las bases sobre las que se
construirá el proyecto.  Este tipo de enfoque se llama ``top-down approach''
%TODO: Cite needed

\subsubsection{tests de aceptación}

Una vez establecidas ciertas bases, se empiezan a definir funcionalidades, casos
de uso.  Aquí es donde entran en juego técnicas de verificación de los casos de
uso.  Se generan documentos ejecutables que consisten en asentar los requisitos
básicos para que un usuario realize una acción y llegue a un resultado final
esperado, dadas unas pre condiciones.  Estos tests, se llaman comúnmente tests
de aceptación, y se encargan de verificar que el software realizado por la
empresa cumple con los requisitos impuestos por el cliente.

El desarrollador puede saber cuando ha acabado de implementar el ``happy path''
de su parte del código cuando su trabajo pasa todos los tests de aceptación que
se han definido con anterioridad.  La ventaja que tienen estos tests frente al
resto de tests que se usan de desarrollo ágil, es que estos tests están
diseñados para que cualquier persona, sea desarrollador, o no, los pueda
definir, crear, y editar.  De esta forma, el desarrollo se mantiene ágil
iteración tras iteración, ya que en caso de haber un cambio entre iteraciones,
únicamente se tienen que ejecutar los tests de aceptación para saber si la
version actual del software cumple con los nuevos requisitos.

La sintaxis estándar de los tests de aceptación es la siguiente:

\begin{lstlisting} Feature: Registro de usuario

  Background: Given la base de datos esta actualizada

  Scenario: Usuario existente inicia sesion Given ya tengo cuenta en la
aplicacion When introduzco mis credenciales correctamente Then inicio sesion en
la aplicacion \end{lstlisting}

De esta forma, se definen las características (``Features'') del producto a
construir.  Cada característica engloba una serie de ``caminos felices'' que
puede seguir el usuario.  La concatenación de acciones que sigue el usuario para
llegar a un estado final se llama ``camino feliz'' (``Scenario'').  Dentro de
cada camino feliz, existen una serie de acciones, representadas por ``Given,
When y Then''.

\begin{itemize} \item{``Given'' representa el estado inicial en el que debe
estar la aplicación} \item{``When'' representa la acción a ejecutar}
\item{``Then'' representa el estado final en el que debe quedar la aplicación.}
\end{itemize}

Este estilo, o formato de escritura se denomina ``Gherkin'' y es fundamental en
la vertiente BDD de desarrollo ágil.

\subsubsection{Desarrollo y testeo del código}

Una vez se ha definido la especificación, los programadores, pueden empezar con
el desarrollo.  Esto asegura que todo el mundo trabaja en sincronía, desde los
clientes, al analista, al jefe de proyecto, al desarrollador.  La existencia de
estos tests hace que cualquier persona involucrada en el proyecto que quiera
revisar la documentación se dirija al mismo sitio.  Esto obliga a que esta
documentación viviente, se mantenga actualizada y al día, entre iteraciones.

A medida que los programadores ahondan en complejidad de desarrollo, se
encuentran con una complejidad cada vez mayor.  Siguiendo con el ejemplo
anterior, del inicio de sesion, se tienen que definir casos mas específicos,
como la forma en la que se conecta la plataforma a la base de datos, el mensaje
de error que aparece cuando el nombre de usuario es incorrecto, y cuando la
contraseña es incorrecta, entre muchos otros.  Para asegurar el correcto
funcionamiento de su código, los desarrolladores deben crear tests para cubrir
cualquier cauce que pueda tomar su código.  De esta forma, pueden delegar la
comprobación de su código a los tests, que van a verificar que la lógica que
están programando es la adecuada, definida previamente.  Una vez se hayan
ejecutado los tests, el programador puede estar seguro de que ha implementado
correctamente el código que se le ha solicitado crear.

Existe un nivel mas detallado de tests.  Estos se llaman tests unitarios.  Su
función es validar el correcto funcionamiento de los componentes específicos
(Funciones, clases, datos) y la forma  que tienen de alterar o crear los datos
necesarios, que, en combinación unos con otros, habilitan la creación de la
lógica de la aplicación.

\subsubsection{Despliegue del código}

Una vez los desarrolladores han terminado la tarea, y el proyecto cumple con las
especificaciones descritas por el cliente, es hora de desplegar el código a
producción.

Al centrarse principalmente en la fase de desarrollo de software, los equipos
que siguen las recomendaciones estipuladas por la metodología ágil,
frecuentemente olvidan la importancia del despliegue en el flujo de desarrollo
del software y el despliegue, deja de ser parte del proceso iterativo, sino que,
al terminar el desarrollo, se despliega el código a producción.

Esta ha sido el modus operandi de la mayoría de equipos que han seguido el
desarrollo ágil.

Esto desemboca en tremendos dolores de cabeza que pueden llegar a prolongarse
durante horas para pasar el código desarrollado en los equipos internos de la
empresa a los equipos destinados para alojar el código en producción o que
directamente pueden llevar a una empresa a la bancarrota en cuestión de minutos.
\cite{seven2014}

\section{De desarrollo ágil, a DevOps}

Llegados a este punto, hay que tener en cuenta que para llevar a cabo cualquier
tipo de proyecto, deben haber como mínimo dos partes implicadas.  El cliente y
la empresa que va a llevar a cabo la vision del cliente.  Estas dos partes
pueden ser tan complejas o simples como se quiera, pero deben existir.  Desde un
proyecto llevado a cabo únicamente por una persona a un proyecto distribuido
separado entre distintas zonas horarias.

La mayor de las responsabilidades del \textbf{cliente} es especificar claramente
los requisitos del producto o idea que tiene en mente.  La responsabilidad de
\textbf{la empresa} es de desarrollar el producto que tiene el ciente en mente
lo mas fielmente posible.  La clave del éxito del desarrollo ágil, basado en
iteraciones incrementales, es que establece canales de comunicación directos,
estrechos e iterativos entre el cliente y la empresa.  (Existen modelos de
desarrollo ágil, en el que se recomienda que el cliente comparta habitación con
la empresa que esta desarrollando su producto)

Aunque, como ya se ha demostrado en empresas no directamente relacionadas con el
mundo del Software como Canon, Honda, o NEC, entre otras, se puede aplicar el
desarrollo ágil a cualquier sector \cite{Hirotaka1986}.  En el mundo del
software, podemos hacer uso de herramientas automatizadas y frameworks para
satisfacer los principales valores del desarrollo ágil.  De esta forma, se logra
crear un proyecto que se mantiene actualizado entre las especificaciones y el
desarrollo.

En 2006, se escribe públicamente por primera vez sobre este proceso, en el
articulo llamado ``The Deployment Production Line'' en el que se describe como
construir una cadena de montaje, en la que el código puede ser desplegado a
producción sencillamente presionando un botón, y revertido, en caso de que algo
fuera mal de igual manera. \cite{10.1109/AGILE.2006.53}

\subsection{Que es DevOps}


La filosofía DevOps nace como respuesta a una falta de comunicación, y creciente
frustración entre equipos dentro de un mismo desarrollo.

Su finalidad es establecer una comunicación constante entre las distintas fases
y equipos de desarrollo, mayoritariamente entre los desarrolladores y los
equipos de operaciones, que se encargan de poner el código que genera el equipo
de desarrolladores a un entorno de producción. Esto no quiere decir que se
excluyan a cualquier otro equipo involucrado en el desarrollo del software, como
el de QA.

El fenómeno DevOps se puede considerar una extension del desarrollo ágil, debido
a que, en su forma mas reducida, acorta las iteraciones del desarrollo de una
fecha especifica, a un commit especifico. \cite{dobra2018} Cada commit pasa por
una serie de comprobaciones o ``jobs'' dentro de un ``pipeline'' para verificar
que efectivamente el código que se quiere añadir a producción es adecuado para
ser desplegado.

Otras de las prioridades de la cultura DevOps es Habilitar la automatización
sobre la documentación, presentar al desarrollador posibilidades (API de testeo
de código), en vez de impedimentos (procesos)

Como consecuencia, se crea un flujo automatizado de Integración Continua y
Despliegue Continuo del código generado iterativamente a lo largo de todo el
proyecto. Este flujo se llama ``Pipeline''.

La filosofía DevOps en la construcción y despliegue de software, por tanto gira
en torno a establecer una serie de procesos y comprobaciones por los que tiene
que pasar cada desarrollo candidato a se desplegado a producción. 

Desde la concepción del proyecto, antes de desarrollar el código, se establecen
los pasos mínimos por los que tiene que pasar cada candidato a producción
(commit) creando así un ``Pipeline'', que es la union de esos pasos
automatizados, de forma que, para que el candidato a producción llegue a
producción, siempre tenga que pasar por estos mismos pasos y comprobaciones.  Ya
sea un apaño en producción, como un desarrollo nuevo.

En la metodología ágil, el desarrollo de una nueva funcionalidad se hace de
forma iterativa e incremental, empezando primero por los tests.  En el enfoque
de la filosofía DevOps, la creación del Pipeline se añade anteriormente a los
tests, para que, una vez se este desarrollando, se puedan revisar los
desarrollos directamente con los tests, de forma automática en el Pipeline una
vez se cumplen una serie de acciones, normalmente, un commit.

Al igual que en el desarrollo ágil, el desarrollo del Pipeline es siempre
constante, este también va creciendo de forma iterativa e incremental, añadiendo
más tests, más pasos, y por tanto revisando y asegurando mayor confianza en el
commit que pasa todo el pipeline.

En esencia, se trata de un desarrollo mas, pero en vez de que su función sea
cumplir un requisito del cliente, su función es asegurar la estabilidad y
calidad del software entregado al cliente. 

``Cada vez que se añade un cambio al servidor de control de versiones, se espera
que pase todos los tests, produzca código funcional y se pueda desplegar a
producción.  Esta es la suposición inicial.  La tarea de un sistema de
integración continua es la de refutar dicha suposición, demostrando que este
candidato a producción no es adecuado para entrar a
producción''\cite{Humble2010}

En el libro ``Accelerate'', de Nicole Forsgren, Jez Humble y Gene Kim, los
autores comentan que la diferencia entre una gran empresa y una pequeña y
mediana empresa, es que las grandes empresas siguen cuatro indicadores de
compromiso muy de cerca, y se han bulto expertas en reducir al máximo estos
mismos.  Los indicadores de compromiso son: el plazo de entrega, la frecuencia
de despliegue, el tiempo medio en restaurar (MTTR) y el porcentaje de cambio vs
error.\cite{Forsgren2018}

Una filosofía DevOps es clave para reducir estos indicadores.  Con el uso de
``Pipelines'' y cambiando la cultura de la empresa, facilitando a los
desarrolladores herramientas y creando flujos y políticas para que estos
indicadores vayan reduciendo el tiempo medio, implica que la empresa el question
tiene un potencial enorme y, por consiguiente una ventaja competitiva sobre sus
competidores.

\section{De DevOps a DevSecOps}

La seguridad en la metodología ágil no ha sido especialmente contemplada durante
la popularización de este tipo de desarrollo, sin embargo, la metodología ágil
destaca por su modularidad, y capacidad de reacción frente a cambios de
requisitos, o nuevas prioridades en el proyecto.  Al poder adaptarse a nuevas
vertientes y filosofías, como la de DevOps, se forman distintas expresiones de
desarrollo ágil, cada una enfatiza en su propia metodología, pero todas parten
de ma misma base. \cite{agilePrinciples}.

\subsection{La seguridad desde un punto de vista clásico}
 
La seguridad en el mundo del software ha estado relegada a un plano inferior
debido a que normalmente no ha sido necesaria para el desarrollo del producto
agenciado.  Al mismo tiempo, durante los años 90 cuando el contenido que
alojaban las paginas web en Internet era mayoritariamente estático, y no estaba
disponible a muchos usuarios, la seguridad era una pieza del puzzle que no
estaba contemplada, ya que los usuarios no tenían forma de interactuar con la
web.  Se parecía, por tanto, a un libro, cuyo contenido esta disponible al
lector una vez ha sido impreso.  La primera vez que se hace referencia a la
nueva web 2.0 es en el articulo titulado ``Fragmented Future'', en el que se
describe el cambio de paradigma que surgirá en la web, a partir de 1999 y se
pasa de tener paginas web estáticas a aplicaciones web, en las que los usuarios
van a interactuar con el contenido. \cite{DiNucci1999} En el desarrollo de estas
aplicaciones web, los desarrolladores implementaban las funcionalidades sin
pensar en como otra gente pudiese romper la aplicacion para sacar provecho del
sistema en el que corre, independientemente de la naturaleza y sector de la
aplicacion.

No obstante, a medida que la tecnología ha avanzado, ha permitido a las empresas
crear modelos de negocio de giraba en torno a Internet.  Un ejemplo bastante
conocido puede ser en caso de Amazon.  Estas empresas ofrecen su servicio, su
valor a través de una pagina web, y los clientes deben estar identificados para
poder obtener los beneficios que ofrecen estas empresas.  Para llevar a cabo el
modelo de negocio anterior, dichas empresas deben tener la capacidad de
identificar a los clientes, por tanto se deben almacenar datos personales.
Estos datos tienen un gran valor para los hackers, que luego pueden venderlos en
el mercado negro.  A medida que Internet se ha extendido, se ha demostrado que
hay gente que ha utilizado dicho factor para sacar un beneficio personal.

El sector del software ha intentado mitigar, o prevenir los ataques de
ciberseguridad mediante iniciativas como por ejemplo programas de recompensa a
la gente que les reporta dichos fallos o vulnerabilidades.  Estos programas se
llaman: ``Threat Bounty Programs'' Por lo general, los requisitos de seguridad
se mantienen separados de otros requisitos de sistema, por tanto, no se integran
en la estrategia general de desarrollo de software. \cite{Flec2003} Esto quiere
decir que los productos que no se han desarrollado con la seguridad en mente,
son mas vulnerables a su explotación.  Las desventajas de esta metodología de
desarrollo son varias, desde una potencial sanción, si se tratan datos sensibles
de usuarios, hasta una explotación de los servicios que ofrece la empresa para
beneficio del individuo.  Cabe destacar la ausencia de cualquier aspecto
relacionado con la seguridad en el paper que popularizo el desarrollo en
cascada. \cite{royce1970}

El sector de la seguridad ha ido creciendo en popularidad desde 1999, en el que
se funda el CVE (Common Vulnerabilities and Exposures).  Entonces, el campo de
la ciberseguridad ha sido relegado al mantenimiento de la aplicacion una vez
lanzada, pero actualmente la industria ha ido identificando la importancia de la
seguridad a nivel de desarrollo y ha ido fomentando esta práctica, con
filosofías como la de DevSecOps para convertirse en una parte fundamental del
proceso de desarrollo de software ágil de hoy en día.

\subsection{Que es DevSecOps}

Siguiendo la inercia del movimiento DevOps, adoptado al inicio por grandes
empresas y seguido de medians y pequeñas empresas, DevSecOps es una extension de
la misma filosofía, reconociendo a la seguridad de la aplicacion y del entorno
en el que se ejecuta como un factor igual de importante en el ciclo de vida del
software.

El propósito de este movimiento es concienciar a los desarrolladores de la
importancia del desarrollo seguro de código, principalmente, cambiando la
cultura de la empresa, para que los desarrolladores tengan el cuenta la
seguridad en el momento en el que desarrollan la aplicacion.

Al igual que en el movimiento DevOps, esto se puede conseguir mediante la
incorporación de ``Jobs'' que revisen el estado del código generado durante el
desarrollo.  Aun así, la mejor forma de seguir esta filosofía es creando una
cultura que la reconozca y respalde, de forma que los desarrolladores no se
molestan al ver que el código tiene vulnerabilidades, sino que evolucionen,
mediante formaciones o reportes de los fallos de su código, a desarrollar de
forma segura.

El ciclo de vida del software, consiste en cinco fases: Planificación, Análisis,
Diseño, Implementación y Mantenimiento.

%TODO: Añade gráfico del ciclo de vida del software.

Anteriormente, la seguridad formaba parte del último paso del ciclo de vida del
desarrollo de software, el mantenimiento.  Como consecuencia, en el momento en
el que se halla una vulnerabilidad de seguridad, o lo que es peor, un agente u
organización externos usan dicha vulnerabilidad para extraer información
sensible de los usuarios de la aplicación, los recursos necesarios para
arreglarla y el tiempo empleado en arreglarla, son muy elevados, al igual que el
estrés y la posible sanción relacionada con la intrusion en el sistema y la Ley
de Protección de Datos.  Este hecho es uno de los principales motivos por el que
las empresas que ya han adoptado el modelo DevOps, están añadiendo ``jobs'' a su
``Pipeline'', encauzando así su filosofía, de DevOps, a DevSecOps y las empresas
que no han adoptado una filosofía DevOps, quieren pasar directamente a
DevSecOps.

\begin{figure} \includegraphics[width=\textwidth]{devsecops.png} \end{figure}

Como podemos observar en el gráfico anterior, el modelo de desarrollo basado en
una filosofía DevSecOps, mueve las comprobaciones de seguridad desde la fase de
mantenimiento, a la fase inicial, contemplando la seguridad como uno más de los
requisitos que se plantean en la fase de planificación y conservando su
importancia durante el resto de fases del proyecto.  En definitiva, la
seguridad, pasa a ser una parte integral del ciclo de vida de desarrollo del
software. 

Para conseguir este propósito, los practicantes de dicha filosofía implementan
``jobs'', o tareas automatizadas relacionadas con la seguridad en el mismo
``Pipeline'', tan característico de DevOps.

En los últimos años, el sector del software ha explotado con empresas como
GitHub, GitLab o GoCD, que proporcionan a los equipos de desarrollo las
herramientas básicas y los pilares fundamentales para que sus clientes adopten
una filosofía DevOps y DevSecOps en la construcción de
software.\cite{Google2019}

Un ejemplo de el uso de la filosofía DevSecOps es el concepto llamado ``Abuser
storie'' \cite{Bor2006} en el que se aplica la seguridad en la fase de
Planificación, a conceptos previos, como los ``User Stories''.

Un user story es una breve descripción escrita por el cliente de no más de tres
líneas sobre la funcionalidad que debe proporcionar la aplicacion.
\cite{XPUserStory} mientras que un ``abuser story'' es una extension de los
``User Stories'' contemplado por un ingeniero de seguridad para poder tomar
decisiones informadas en base al estado de la seguridad en el momento de su
definición. \cite{Bor2006}

Esta forma de re plantearse la seguridad es exactamente el cambio de paradigma
que lleva a la filosofía DevSecOps.


\chapter{Implementación del modelo DevSecOps}

El contenido de esta sección, define el problema al que nos enfrentamos,
analizando los principales problemas con el \textit{\gls{pipeline}} actual en
Scalefast.  Posteriormente, definiremos los objetivos a los que queremos llegar,
detallando la razón por la cual se piensa que es el camino correcto en este
momento para Scalefast.
%TODO: eliminar objetivos de la fase y re formular
Una vez definidos los objetivos, vamos a detallar el ciclo de desarrollo y
puesta en producción de uno de nuestros objetivos detallados anteriormente.
Para terminar, se desarrollará la conclusion a la que se ha llegado tras
realizar este análisis del estado actual del \textit{\gls{pipeline}} y su
subsecuente esfuerzo por evolucionarlo de una forma paralela a la dirección que
está llevando a cabo la industria del software ágil.

\section{Análisis del estado actual del pipeline} %DEFINICIÓN DEL PROBLEMA

En Scalefast, se sigue un modelo de desarrollo similar al descrito
anteriormente, en la metodología ágil.  Actualmente, la empresa tiene un
\textit{\gls{pipeline}} de integración continua, en el que se pasan varios
``jobs'' o pruebas automáticas en cada commit vinculado a un \acrfull{mr} que se
sube al repositorio de código.  Un \textit{\gls{pipeline}} es una pieza clave
del ciclo de vida del software, responsable de asegurar la calidad de los
desarrollos que integramos a producción y que por ende, definen la calidad de la
empresa.  Dicho esto, es importante recordar que el desarrollo del
\textit{\gls{pipeline}} debe seguir siempre el modelo de desarrollo de software
que el mismo promueve, el desarrollo iterativo e incremental comprendido esto,
es evidente que su desarrollo no tiene fin.  En vista de la progresión de
Scalefast, debemos evolucionar el pipeline, haciendo que tienda mas hacia la
integración y despliegues continuous, centrando ademas el desarrollo de software
en la seguridad.  A medida que Scalefast ha crecido tanto en clientes, como en
empleados y proyectos que abarcamos, se han identificado una serie de problemas
con el \textit{\gls{pipeline}} que usamos actualmente.

\section{Definición de procesos}

Como se ha definido previamente, el objetivo principal de este trabajo es
reciclar el \textit{\gls{pipeline}} que tenemos en funcionamiento actualmente en
Scalefast por uno mas actual, que sea capaz de proporcionar a los equipos las
herramientas y entornos que necesitan en el momento en el que las necesitan, de
esta forma, podemos reducir el tiempo de desarrollo, aumentando ademas la
calidad y seguridad del software que sacamos a producción.  De esta forma,
Scalefast adquiere une ventaja competitiva injusta.

\subsection{Documentación y decision de procesos}

Para poder implementar el \textit{\gls{pipeline}} en forma de proceso en una
empresa medianamente grande, hace falta la cooperación de varios responsables de
equipo.  Este \textit{\gls{pipeline}} va a afectar tanto a los equipos de QA, a
los desarrolladores, y al equipo de Operaciones y Arquitectura.  En definitiva,
toda la parte técnica de la empresa se va a ver afectada ya que se debe tener en
cuenta el punto de vista de todos los equipos mencionados anteriormente para que
el resultado generado aporte beneficios a todos los equipos implicados.

\subsubsection{Análisis del \textit{\gls{pipeline}} desde el punto de vista de
seguridad}

Desde el equipo de seguridad, se ha decidido crear un \textit{\gls{pipeline}}
conceptual, con todas las comprobaciones y herramientas que se han considerado
oportunas.  A partir de este esquema, comparamos el estado actual de nuestro
\textit{\gls{pipeline}} y definimos una serie de perfiles, procesos y
desarrollos necesarios para llevarlo desde el punto actual, a la mejor version
del pipeline, en términos de seguridad que podamos.  Posteriormente creamos el
plan de desarrollo, exactamente igual que si se tratase de un proyecto de un
cliente externo.

Una vez se tenia claro que el \textit{\gls{pipeline}} se iba a actualizar, cada
equipo ha confeccionado una lista de acciones que quieren que ejecute el
\textit{\gls{pipeline}} para verificar que el código generado cumple con los
requisitos de calidad propios de Scalefast.

% Definir el estado del pipeline actual ANTES vs DESPUÉS
Una de las razones por las que se decide actualizar el \textit{\gls{pipeline}}
es porque se quiere promover el desarrollo seguro desde el inicio del ciclo de
vida del proyecto.  Actualmente, la forma que tenemos de revisar la seguridad
del proyecto que creamos es mediante un análisis manual de seguridad una vez ha
terminado el desarrollo del proyecto.  Los inconvenientes heredados de seguir
esta metodología de desarrollo actualmente son que en el momento en el que se
encuentra una vulnerabilidad, el tiempo que se tarda en desarrollar una solución
para el problema incluye el tiempo en analizar el problema, la asignación al
desarrollador responsable de ese fallo de seguridad, o a uno nuevo, el
planteamiento de la solución del problema, el desarrollo de la misma, la
revision, para asegurar que el fallo esta neutralizado y la subsecuente revision
para asegurar que no se han añadido fallos de regresión o nuevas
vulnerabilidades.  En cambio, si se incluye un analista de seguridad desde la
fase de concepción del proyecto, durante el ciclo de vida del desarrollo y
ademas, se crean \textit{\gls{job}s} para analizar la parte mas automatizarle
del desarrollo, entonces, se reduce considerablemente la probabilidad de
encontrar un fallo en los últimos pasos del desarrollo del proyecto.  Este es el
punto al que queremos llegar.

\section{DevSecOps en uso}

Recientemente, Mike Ensor y Drew Stevens publicaron un ``paper'' titulado
``Shifting left on security''.  En él, explican las diferentes fases de el ciclo
de vida del software, y como añadir elementos de seguridad a lo largo del
``pipeline''.  Desde la planificación del proyecto, hasta el despliegue del
mismo.  Algunas de las técnicas que recomiendan en su ``paper'' tienen gran
importancia, como el uso de registros privados, cuyas imágenes y recursos estén
identificadas y sean verificadas a la hora de desplegarse, análisis estático y
dinámico del código que se escribe de forma automatizada, despliegue del mismo
binario, y que solamente pueda ser desplegado si es firmado de forma
criptográfica por el responsable de despliegues.\cite{Ensor2021}

En esta sección, se describen mas en detalle dichas herramientas y se describe
la implementación de alguna de ellas en el \textit{\gls{pipeline}} de Scalefast.

\clearpage

\printglossary[type=\acronymtype]

\printglossary

%----------
%	Bibliography
%----------	

\clearpage
\addcontentsline{toc}{chapter}{Bibliografía}

\printbibliography


%---------- Appendix ----------	

% If your work includes Appendix, you can uncomment the following lines
%\chapter* {Appendix x} \pagenumbering{gobble} % Appendix pages are not numbered

\end{document}

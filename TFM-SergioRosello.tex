%----------
%   WARNING
%----------

% This Guide contains Library recommendations based mainly on APA and IEEE styles, but you must always follow the guidelines of your TFG Tutor and the TFG regulations for your degree.

% THIS TEMPLATE IS BASED ON THE APA STYLE 


%----------
% DOCUMENT SETTINGS
%----------

\documentclass[12pt]{report} % font: 12pt

% margins: 2.5 cm top and bottom; 3 cm left and right
\usepackage[
a4paper,
vmargin=2.5cm,
hmargin=3cm
]{geometry}

% Paragraph Spacing and Line Spacing: Narrow (6 pt / 1.15 spacing) or Moderate (6 pt / 1.5 spacing)
\renewcommand{\baselinestretch}{1.15}
\parskip=6pt

% Color settings for cover and code listings 
\usepackage[table]{xcolor}
\definecolor{azulUC3M}{RGB}{0,0,102}
\definecolor{gray97}{gray}{.97}
\definecolor{gray75}{gray}{.75}
\definecolor{gray45}{gray}{.45}

% In the template we include the file OUTPUT.XMPDATA. You can download that file and include the metadata that will be incorporated into the PDF file when you compile the memoria.tex file. Then upload it back to your project. 
\usepackage[a-1b]{pdfx}

% LINKS
\usepackage{hyperref}
\hypersetup{colorlinks=true,
	linkcolor=black, % links to parts of the document (e.g. index) in black
	urlcolor=blue} % links to resources outside the document in blue

% MATH EXPRESSIONS
\usepackage{amsmath,amssymb,amsfonts,amsthm}

% Character encoding
\usepackage{txfonts} 
\usepackage[T1]{fontenc}
\usepackage[utf8]{inputenc}

% English settings
\usepackage[spanish]{babel} 
\usepackage[babel, spanish=spanish]{csquotes}
\AtBeginEnvironment{quote}{\small}

% Footer settings
\usepackage{fancyhdr}
\pagestyle{fancy}
\fancyhf{}
\renewcommand{\headrulewidth}{0pt}
\rfoot{\thepage}
\fancypagestyle{plain}{\pagestyle{fancy}}

% DESIGN OF THE TITLES of the parts of the work (chapters and epigraphs or sub-chapters)
\usepackage{titlesec}
\usepackage{titletoc}
\titleformat{\chapter}[block]
{\large\bfseries\filcenter}
{\thechapter.}
{5pt}
{\MakeUppercase}
{}
\titlespacing{\chapter}{0pt}{0pt}{*3}
\titlecontents{chapter}
[0pt]                                               
{}
{\contentsmargin{0pt}\thecontentslabel.\enspace\uppercase}
{\contentsmargin{0pt}\uppercase}                        
{\titlerule*[.7pc]{.}\contentspage}                 

\titleformat{\section}
{\bfseries}
{\thesection.}
{5pt}
{}
\titlecontents{section}
[5pt]                                               
{}
{\contentsmargin{0pt}\thecontentslabel.\enspace}
{\contentsmargin{0pt}}
{\titlerule*[.7pc]{.}\contentspage}

\titleformat{\subsection}
{\normalsize\bfseries}
{\thesubsection.}
{5pt}
{}
\titlecontents{subsection}
[10pt]                                               
{}
{\contentsmargin{0pt}                          
	\thecontentslabel.\enspace}
{\contentsmargin{0pt}}                        
{\titlerule*[.7pc]{.}\contentspage}  


% Tables and figures settings
\usepackage{multirow} % combine cells 
\usepackage{caption} % customize the title of tables and figures
\usepackage{floatrow} % we use this package and its \ ttabbox and \ ffigbox macros to align the table and figure names according to the defined style.
\usepackage{array} % with this package we can define in the following line a new type of column for tables: custom width and centered content
\newcolumntype{P}[1]{>{\centering\arraybackslash}p{#1}}
\DeclareCaptionFormat{upper}{#1#2\uppercase{#3}\par}
\usepackage{graphicx}
\graphicspath{{imagenes/}} % images folder

% Table layout for social sciences and humanities
\captionsetup*[table]{
	justification=raggedright,
	labelsep=newline,
	labelfont=small,
	singlelinecheck=false,
	labelfont=bf,
	font=small,
	textfont=it
}

% Figure layout for social sciences and humanities
\captionsetup[figure]{
	%name=Figura,
	singlelinecheck=off,
	labelsep=newline,
	font=small,
	labelfont=bf,
	textfont=it
}
\floatsetup[figure]{
    style=plaintop,
    heightadjust=caption,
    footposition=bottom,
    font=small
}

% Figures and tables footnote layout 
\captionsetup*[floatfoot]{
    footfont={small, up}
}

% FOOTNOTES
\usepackage{chngcntr} % continuous numbering of footnotes
\counterwithout{footnote}{chapter}

% CODE LISTINGS 
% support and styling for listings. More information in  https://es.wikibooks.org/wiki/Manual_de_LaTeX/Listados_de_código/Listados_con_listings
\usepackage{listings}

% Custom listing
\lstdefinestyle{estilo}{ frame=Ltb,
	framerule=0pt,
	aboveskip=0.5cm,
	framextopmargin=3pt,
	framexbottommargin=3pt,
	framexleftmargin=0.4cm,
	framesep=0pt,
	rulesep=.4pt,
	backgroundcolor=\color{gray97},
	rulesepcolor=\color{black},
	%
	basicstyle=\ttfamily\footnotesize,
	keywordstyle=\bfseries,
	stringstyle=\ttfamily,
	showstringspaces = false,
	commentstyle=\color{gray45},     
	%
	numbers=left,
	numbersep=15pt,
	numberstyle=\tiny,
	numberfirstline = false,
	breaklines=true,
	xleftmargin=\parindent
}

\captionsetup*[lstlisting]{font=small, labelsep=period}
 
\lstset{style=estilo}
\renewcommand{\lstlistingname}{\uppercase{Código}}


% REFERENCES 

% APA bibliography setup
\usepackage[style=apa, backend=biber, natbib=true, hyperref=true, uniquelist=false, sortcites]{biblatex}

\addbibresource{referencias.bib} % The references.bib file in which the bibliography used should be

% Caption package, for use of subfigures.
\usepackage{subfig}


%-------------
%	DOCUMENT
%-------------

\begin{document}

\pagenumbering{roman} % Roman numerals are used in the numbering of the pages preceding the body of the work.
	
%----------
%	COVER
%----------	
\begin{titlepage}
	\begin{sffamily}
  \begin{figure}%
    \raggedleft
    \subfloat{{\includegraphics[width=3cm]{UNED.jpg} }}%
    \hspace*{\fill}
    \subfloat{{\includegraphics[width=6cm]{Scalefast.jpg} }}%
\end{figure}
	\begin{center}
		\vspace{2.5cm}
		\begin{Large}
			Master en Ciberseguridad\\			
			 2020-2021\\
			\vspace{2cm}		
			\textsl{Trabajo de final de Master}
			\bigskip
			
		\end{Large}
		 	{\Huge ``Definición y puesta en funcionamiento del sistema operacional de seguridad en el ciclo de desarrollo/despliegue (DevSecOps)''}\\
		 	\vspace*{0.5cm}
	 		\rule{10.5cm}{0.1mm}\\
			\vspace*{0.9cm}
			{\LARGE Sergio Roselló Morell}\\ 
			\vspace*{1cm}
		\begin{Large}
			Rafael Pastor Vargas\\
			David Aracil Cofrade\\
			Madrid, a \today \\
		\end{Large}
	\end{center}
	\vfill
	\color{black}
	\fbox{
	\begin{minipage}{\linewidth}
    	\textbf{AVOID PLAGIARISM}\\
    	\footnotesize{The University uses the \textbf{Turnitin Feedback Studio} for the delivery of student work. This program compares the originality of the work delivered by each student with millions of electronic resources and detects those parts of the text that are copied and pasted. Plagiarizing in a TFM is considered a  \textbf{\underline{Serious Misconduct}}, and may result in permanent expulsion from the University.}\end{minipage}}

	% IF OUR WORK IS TO BE PUBLISHED UNDER A CREATIVE COMMONS LICENSE, INCLUDE THESE LINES. IS THE RECOMMENDED OPTION.
	\noindent\includegraphics[width=4.2cm]{creativecommons.png}\\ % Creative Commons Logo
    \footnotesize{This work is licensed under Creative Commons \textbf{Attribution – Non Commercial – Non Derivatives}}
	
	\end{sffamily}
\end{titlepage}

\newpage % blank page
\thispagestyle{empty}
\mbox{}




%----------
%	ABSTRACT AND KEYWORDS 
%----------	
\renewcommand\abstractname{\large\bfseries\filcenter\uppercase{Sumario}}
\begin{abstract}
\thispagestyle{plain}
\setcounter{page}{3}
	
Inicio del movimiento DevOps, el 2010, con el desarrollo agil\\
Adopcion de la cultura agil por silicon valley y resto de munto\\
Implementacion de seguridad dentro del desarrollo agil\\
Uso de Pipelines, para desarrollar SW con la seguridad en mente\\
Caso especifico de la implementacion en Scalefast\\
\vfill
\end{abstract}
	
\newpage % Blank page
\renewcommand\abstractname{\large\bfseries\filcenter\uppercase{Abstract}}
\begin{abstract}
\thispagestyle{plain}
\setcounter{page}{5}

Start of Agile SW development, and DevOps culture\\
Adoption of culture by Silicon Valley and subsequent popularization\\
Security inside agile development?\\
Use of Pipelines in security-focused Agile development\\
Specific case with Scalefast\\

\textbf{Palabras clave:} % add the keywords
	
\vfill
\end{abstract}
\newpage % Blank page
\thispagestyle{empty}
\mbox{}


%----------
%	Dedication
%----------	
\chapter*{Agradecimientos}

\setcounter{page}{7}
	
Agradecer principalmente este trabajo a:

\begin{itemize}
  \item{Familia}
  \item{Tutores}
  \item{UNED}
  \item{Scalefast}
\end{itemize}

		
	\vfill
	
	\newpage % blank page
	\thispagestyle{empty}
	\mbox{}
	




%----------
%	TOC
%----------	

%--
% TOC
%-
\tableofcontents
\thispagestyle{fancy}

\newpage % blank page
\thispagestyle{empty}
\mbox{}




%--
% List of figures. If they are not included, comment the following lines
%-
\listoffigures
\thispagestyle{fancy}

\newpage % blank page
\thispagestyle{empty}
\mbox{}




%--
% List of tables. If they are not included, comment the following lines
%-
\listoftables
\thispagestyle{fancy}

\newpage % blankpage
\thispagestyle{empty}
\mbox{}





%----------
%	THESIS
%----------	
\clearpage
\pagenumbering{arabic} % numbering with Arabic numerals for the rest of the document.	

\chapter{Introducción}

Esta es la pequeña introducción de la tesis, a ver si funciona como me gastaría a mi :)
\cite{kim}

%----------
%	Bibliography
%----------	

\clearpage
\addcontentsline{toc}{chapter}{Bibliografía}

\printbibliography


%----------
%	Appendix
%----------	

% If your work includes Appendix, you can uncomment the following lines
%\chapter* {Appendix x}
%\pagenumbering{gobble} % Appendix pages are not numbered

\end{document}
